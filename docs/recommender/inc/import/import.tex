

\section{Import from data}
\label{sec:importfromdata}
As input data I used the given DamenBlusen.txt-file.
Unfortunately I had some trouble importing it.\\
The structure of the different lines in the file is inconsistent.
\begin{itemize}
    \item Some words (which will later be used as term to describe a product) are merely synonym to for others (such as: 'whisper white' and 'white', 'eggnog' and 'beige')
    \item Or even worse are translations of the same attribute (such as 'white' and 'wei\"zs')
    \item Switches between 'in Gr\"o\"sen' and 'bei Gr\"o\"sen'
\end{itemize}
Currently there are three items with a price, which can not be imported.\\
Items without a price are being ignored!


\section{Implementation}
Using of a regex-expression to get the bits of information out of DamenBlusen.txt.
That's more or less reliable (see section above\ref{sef:importfromdata}).
Following attributes can be reliably excluded from a DamenBlusen-entry:
\begin{itemize}
    \item image name/url
    \item brand
    \item type of cloth
    \item price
    \item colours
    \item materials
\end{itemize}

\medskip
Even though not all of these attributes can be used as term for a vector.
\sout{My guess is that 'price' must be dropped, since they may vary a lot for each product.}
For instance 'image\_name' can not be used as term, since every entry has its very own distinctive image.

\subsection{Creation of document vectors}

\begin{table}[h]
    \begin{center}
    \rowcolors{0}{LightGray}{White}
    \begin{tabular}{ l l l l l }
        \rowcolor{Gray}
        \textbf{clothing\_id}   & \textbf{name} & \textbf{brand}    & \textbf{colour}   & \textbf{product}\\
        1                       & bluse\_1      & N\"umph           & 2399             & Bluse\\
        2                       & bluse\_2      & s.Oliver          & blue              & Bluse\\
        3                       & bluse\_3      & N\"umph           & red               & Bluse\\
        3                       & bluse\_3      & N\"umph           & 2399             & Bluse\\
        4                       & bluse\_4      & s.Oliver          & blue              & Bluse\\
    \end{tabular}
    \caption{Example of products within a database}
    \end{center}
\end{table}

\bigskip

\subsection{Term frequency (tf)}

\begin{quote}
    $t$ = Term (a part of a search query)\\
    $d$ = Document (the document is a collection of terms that describes a product)
\end{quote}
The weight is equal to the number of occurences of the term $t$ in document $d$.

\noindent
\begin{table}[h]
    \begin{center}
    \rowcolors{0}{LightGray}{White}
    \begin{tabular}{ l l l l l l l }
        \rowcolor{Gray}
        \textbf{clothing\_id}   & \textbf{N\"umph} & \textbf{s.Oliver}   & \textbf{2399}    & \textbf{blue} & \textbf{red} & \textbf{Bluse}\\
        1                       & 1                 & 0                 & 1                 & 0             & 0             & 1\\
        2                       & 0                 & 1                 & 0                 & 1             & 0             & 1\\
        3                       & 1                 & 0                 & 1                 & 0             & 1             & 1\\
        4                       & 0                 & 1                 & 0                 & 1             & 0             & 1\\
    \end{tabular}
    \caption{Example vector of a product}
    \end{center}
\end{table}


\subsection{Inverse document frequency (idf)}
\begin{quote}
    $cf_t$ = collection frequency (occurences of a term $t$ in all documents)\\
    $df_t$ = document frequency (occurences of a term $t$ in a given document)
\end{quote}
In order to scale down the weight/impact of a term that occurs in nearly every document (such as 'Bluse') we can use the so called inverse document frequency.\\
The intent of idf is to scale down the tf weight of a term by a factor that grows with its collection frequency.\\

$idf_t = log_{10}\frac{N}{df_{t}}$

\bigskip
$N = \displaystyle\sum_{d \in documents}{d} = 4$

\bigskip
\begin{table}[h]
    \begin{center}
    \rowcolors{0}{LightGray}{White}
    \begin{tabular}{ l l l l l l l }
        \rowcolor{Gray}
        \textbf{$df_{Numph}$} & \textbf{$df_{s.Oliver}$} & \textbf{$df_{2399}$} & \textbf{$df_{blue}$} & \textbf{$df_{red}$} & \textbf{$df_{Bluse}$}\\
        2                     & 2                        & 2                     & 2                    & 1                   & 4\\
    \end{tabular}
    \caption{document frequency}
    \end{center}
\end{table}

\bigskip
\begin{table}[h]
    \begin{center}
    \rowcolors{0}{LightGray}{White}
    \begin{tabular}{ l l l l l l l }
        \rowcolor{Gray}
            \textbf{$idf_{Numph}$} & \textbf{$idf_{s.Oliver}$} & \textbf{$idf_{2399}$} & \textbf{$idf_{blue}$} & \textbf{$idf_{red}$} & \textbf{$idf_{Bluse}$}\\
        0.3                    & 0.3                       & 0.3                    & 0.3                   & 0.6                  & 0
    \end{tabular}
    \caption{idf-weights}
    \end{center}
\end{table}


\subsection{tf-idf weighting}
We combine the difinition of term frequency (tf) and inverse document frequency (idf), to produce a composite weight for each term in a document.\\
The tf-idf weighting scheme assigns to a term $t$ a weight in a document $d$ given by\\

$tf-idf_{t,d}= tf_{t,d} \times idf_t$

\bigskip
\begin{table}[h]
    \begin{center}
    \rowcolors{0}{LightGray}{White}
    \begin{tabular}{ l l l l l l l }
        \rowcolor{Gray}
        \textbf{clothing\_id}   & \textbf{$tf-idf_{Numph}$} & \textbf{$tf-idf_{s.Oliver}$}   & \textbf{$tf-idf_{2399}$}    & \textbf{$tf-idf_{blue}$} & \textbf{$tf-idf_{red}$} & \textbf{$tf-idf_{Bluse}$}\\
        1                       & 0.3              & 0                   &  0,3              & 0             & 0            & 0\\
        2                       & 0                & 0.3                 & 0                 & 0.3           & 0            & 0\\
        3                       & 0.3              & 0                   & 0.3               & 0             & 0.6          & 0\\
        4                       & 0                & 0.3                 & 0                 & 0.3           & 0            & 0\\
    \end{tabular}
    \caption{tf-idf weights}
    \end{center}
\end{table}










