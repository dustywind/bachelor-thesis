

\section{Motivation}
Currently the apparel industry is not one of the healthiest sectors in the industry.
That's due to conflicts in Middle East and Ukraine, as well as problems within the European Union.\citep[p. 4]{alten:14}
The lack of growth within the clothing industry and even a decline in sales, however, do not mean that the diversity or quantity of products on the market became any less.
There are still way more products than a average human can handle.\\
In short: Consumers still have a wide variety of products from which they can choose.
And anywhere, where the count of choices one can take is to high to grasp, it comes in handy to have something that helps making decisions.\\
One of these tools are so called recommendation systems. The concept of those will be explained in detail later.


\subsection{Choice overload}
Due to the nearly unlimited number of choices one can make, regarding the variety of products, many customers have trouble finding the products they like best.
This problem is commonly known as "choice overload".\citep[p. 454]{stanton:12}
One domain the problem has been noticed is the clothing industry.
The reason for choice overload range from the massive amount of clothing one can choose from to some social aspects.
There are many possible reasons why customer struggle with decision-making\citep[p. 454]{stanton:12} - however, for this project the scope is limited to possible approaches which solve the problem.


\subsection{The recommendation system}
One of the projects goals is to implement a fully functional recommendation system.
The core component of the system is the algorithm that generates suggestions.
For this specific project Rocchio's relevance feedback algorithm will be used.
The algorithm will work on product-data taken from an online-shop in order to simulate the real world as good as possible.
A benefit of using apparel as product is the already mentioned huge amount of data one can use.
Zalando, a German online shop specialised on clothing for instance, offers about 150,000 products from 1,500 different brands.\citep{visser:14}
Nevertheless the algorithm and all components which do preliminary work have to be capable of operating on any data that can be shaped into a proper form.


%algorithmus klappt eigentlich immer.
%aber kleidungsindustrie ist attraktiv
%zalandoo anzahl der kleidungsstuecke
%amazon...
%...


