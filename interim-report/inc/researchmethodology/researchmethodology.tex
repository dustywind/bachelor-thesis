


\section{Research methodology}
%- Forschungsmethodik/Vorgehen
%   -- vorgehen mit beispiel der Forschungsfragen

This section discusses the approach for helping customers coming to a decision using a recommender system.\\
At first we have to define the tasks a recommendation system has to solve in order to be effective.
We will also peak on the variety of approaches to generate suggestions such as "content-based", "knowledge-based", etc.\\
In order to test a recommender system one will be implemented, based on Rocchio's information retrieval algorithm, and combined with an online-shop - even though the online shop will only concentrate on the core-component of presenting products.\\
A characteristic of this algorithm is the way it refines the set of recommendations every time a user gives more information about his preferences.\citep[p. 92]{lops:11}
The process of learning about the user can both be implicit, or explicit.
Learning based on implicit behaviour will be implemented in the first version.
Depending on the time schedule an alternative implementation with explicit user feedback is possible.



