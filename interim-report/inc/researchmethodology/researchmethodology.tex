


\section{Research methodology}
%- Forschungsmethodik/Vorgehen
%   -- vorgehen mit beispiel der Forschungsfragen

The approach to help customers coming to a decision is a recommender system.

At first we have to define the tasks a recommendation system has to solve to be effective.
We will also peak on the variety of approaches to generate suggestions such as "content-based", "knowledge-based", etc.

In order to test a recommender system we will implement our own, based on Rocchios information retrieval algorithm.
A characteristic of this algorithm is the way, it refines the set of recommendations every time a user gives more information about his preferences.\citep[p. 92]{lops:11}

The process of learning about the user can both be implicit, or explicit.
We will implement learning based on implicit behaviour.
Depending on the time schedule an alternative implementation with explicit user feedback is possible.

% KEINE Content analysierung
% oder wie auch immer das heisst




This is a "content-based" approach.

As testing environment for the recommendation system we will built up an online-shop that generates suggestions for each user individually.
In order to find anything that matters, one can make use of computing power.
A possible approach help customers coming to a decision is provided by the so called Rocchio Information Retrieval Algorithm.

