

\section{Research question}
When designing a recommendation system there are some questions, such as the requirements to a recommender system, left to solve.
In the following requirements on a recommender system will be defined.
The bachelor thesis aims to solve the issue whether the Rocchio relevance feedback algorithm is capable of satisfying all requirements or not.

\begin{itemize}
    \item \textbf{requirements to an recommender system}\\
        What are the key requirements a recommender system has to fulfill?

    \item \textbf{different approaches}\\
        There are many approaches to generate automated recommendations such as "content-based", "collaborative filtering", "demographic", "knowledge-based", "community-based".\citep[p.~10-12]{ricci:11}\\
        Also the source of user feedback can vary from implicit to explicit feedback.\citep[p. 76]{lops:11}\\
        How do they work and which approach does rocchio's algorithm take?\\
        Which of these approaches is best suited to generate clothing-recommendations based on implicit feedback?

    \item \textbf{computer representation}\\
        How can products be represented on a computer?\\
        Can the Rocchio algorithm be adjusted at runtime?\\
        % fuer neue nutzer schnell lernend
        % am anfang einer neuen Mode-Saison schnell lernend
        How can possible recommendations be found?

    \item \textbf{testing}\\
        Are the results of the algorithm (tested in a self-built online-shop) satisfying?

    \item \textbf{outlook}\\
        How can the rocchio algorithm be implemented in a "real" (online-)shop?
        
\end{itemize}


%Wie kann ein system konzeptioniert und implementiert zu werden um efektive kleidungsempfehlung basierend auf implizitem feedback zu geben?


%- Forschungsmethodik/Vorgehen
%   -- vorgehen mit beispiel der Forschungsfragen


