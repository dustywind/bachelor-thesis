

\section{Research question}
In the following are requirements on a recommender system defined.
The bachelor thesis aims to solve the issue, whether the Rocchio relevance feedback algorithm is capable of satisfying all requirements or not.
Also different approaches will be combined - also with regard of the rocchio algorithm.


\begin{itemize}
    \item \textbf{requirements to an recommender system}\\
        What are the key requirements a recommender system has to fulfill?

    \item \textbf{different approaches}\\
        There are many approaches to generate automated recommendations such as "content-based", "collaborative filtering", "demographic", "knowledge-based", "community-based".\citep[p.~10-12]{ricci:11}\\
        How do they work and which approach does rocchio's algorithm take?\\
        Which of these approaches is best suited to generate clothing-recommendations based on implicit feedback?

    \item \textbf{computer representation}\\
        How can products be represented on a computer?\\
        Can the rocchio algorithm be adjusted at runtime?\\
        How can possible recommendations be found?

    \item \textbf{testing}\\
        Are the results of the algorithm (tested in a self-built online-shop) satisfying?

    \item \textbf{outlook}\\
        How can the rocchio algorithm be implemented in a shop?
        
\end{itemize}



%Wie kann ein system konzeptioniert und implementiert zu werden um efektive kleidungsempfehlung basierend auf implizitem feedback zu geben?




%- Forschungsmethodik/Vorgehen
%   -- vorgehen mit beispiel der Forschungsfragen



