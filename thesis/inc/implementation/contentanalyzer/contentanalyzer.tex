
\subsection{Content Analyzer}
\label{sec:content-analyzer}
\index{Content Analyzer}
For this project the data describing the products, offered by an online shop have been semi-structured.
It was a text file where each line described a product.
An example is given in listing~\ref{lst:product-data}.

\begin{lstlisting}[caption={Example product data},label={lst:product-data}
    ,keywordstyle=\color{black}
    ,commentstyle=\itshape\color{black}
    ,identifierstyle=\color{black}
    ,stringstyle=\color{black}
]
ImgURL Brand Product Price Shoulder_Width Model_Length Collar_Type Material
http://i1.ztat.net/large/4E/M2/1E/00/0K/11/4EM21E000-K11@4.jpg Emoi en Plus Bluse - dazzling blue 24,95 °(\euro{})° 50 cm 70 cm bei Gr°(\"{o}\ss{})°e 44 Rundhals 100% Polyester
http://i2.ztat.net/large/NA/52/1D/03/NA/11/NA521D03N-A11@3.jpg NAF NAF WENT - Bluse - ecru/noir 38,95 °(\euro{})° 55 cm bei Gr°(\"o{}\ss{})°e S Rundhals 64% Viskose, 22% Baumwolle, 10% Modal, 4% Polyamid
\end{lstlisting}

%\noindent
%Since the structure of the input was known, it was possible to filter out all relevant product information without using too fancy IR methods.
%With regular expressions all relevant informations such as the product\_image-url, brand, product, price, collar type and material have been extracted and stored in a database.
From each line describing a product information such as the image URL and materials have been extracted.
Except from the image URL every word separated by whitespace has been interpreted as term.
Since a RS could theoretically handle any kind of item afar from products, a distinction has been made between documents in general and products.
The relation between a product and a document has been illustrated in figure~\ref{fig:ertermdocumentassignment}.
\begin{figure}[h]
    \center
    \includegraphics[scale=0.5]{inc/implementation/contentanalyzer/er_term_document_assignment}
    \caption{ER diagram of documents, products and associated terms}
    \label{fig:ertermdocumentassignment}
\end{figure}
Since the relation between \textit{Product} and \textit{Term} is N to N, an intermediate table is necessary when transforming the ER-diagram into a relational model.
Therefore the table \textit{TermDocumentAssigner} has been introduced and the relational model will look as follows:

\begin{quote}
    \textbf{Document}{(\underline{document\_id})}\\
    \textbf{Product}{(\underline{document\_id[Document]}, image\_name)}\\
    \textbf{Term}{(\underline{term\_id}, name)}\\
    \textbf{TermDocumentAssigner}{(\underline{document\_id[Document], term\_id[Term]}, count)}\\
\end{quote}

Because \textit{TermDocumentAssigner} will be very often used to query all terms of a document, it might be useful to create an index on \textit{document\_id}.
Implicitly there will also be one on the combined primary key \textit{document\_id, term\_id}.
The tables, implemented by the RS built for this thesis, filled with example data may look as in table~\ref{tab:tablestermdocumentproduct}.
Table~\ref{tab:tablestermdocumentproduct} will serve as resource for terms and documents to illustrate subsequent examples.

\begin{table}

    \center

    % Document
    \rowcolors{1}{\dustRowFirst}{\dustRowSecond}
    \begin{tabular}{ l }
        \rowcolor{\dustRowHead}
        \textbf{Document}\\\hline
        document\_id\\\hline
        1\\
        2\\
        3\\
    \end{tabular}
    \quad
    % Product
    \rowcolors{1}{\dustRowFirst}{\dustRowSecond}
    \begin{tabular}{ l | l }
        \rowcolor{\dustRowHead}
        \multicolumn{2}{ c }{\textbf{Product}}\\\hline
        document\_id    & image\_name\\\hline
        1               & image\_1.png\\
        2               & image\_2.png\\
        3               & image\_3.png\\
    \end{tabular}

    %~\\
    \hfill\\

    %TermDocumentAssigner
    \rowcolors{1}{\dustRowFirst}{\dustRowSecond}
    \begin{tabular}{ l | l | l }
        \rowcolor{\dustRowHead}
        \multicolumn{3}{c}{\textbf{TermDocumentAssigner}}\\\hline
        document\_id    & term\_id  & count\\\hline
        1               & 1         & 1\\
        1               & 2         & 1\\
        1               & 4         & 1\\
        2               & 1         & 1\\
        2               & 3         & 1\\
        2               & 7         & 1\\
        3               & 4         & 1\\
        3               & 5         & 1\\
        3               & 6         & 1\\
    \end{tabular}
    \quad
    % Term
    \rowcolors{1}{\dustRowFirst}{\dustRowSecond}
    \begin{tabular}{ l | l }
        \rowcolor{\dustRowHead}
        \multicolumn{2}{ c }{\textbf{Term}}\\\hline
        term\_id        & name\\\hline
        1               & blouse\\
        2               & blue\\
        3               & polyester\\
        4               & cotton\\
        5               & green\\
        6               & trouser\\
        7               & white\\
    \end{tabular}
    \caption{Table layout defined by figure~\ref{fig:ertermdocumentassignment}}
    \label{tab:tablestermdocumentproduct}
\end{table}

%\noindent
As already mentioned before, Rocchio's algorithm works best with tf-idf vectors.
With the theory described in section~\ref{sec:tfidf} the next big step is to show how vectors for this project have been built.
As a short reminder: all products are described through their terms.

In order to build tf-idf vectors, one also has to build term frequency-, as well as inverse document frequency vectors - these are the preconditions.
Since these tasks are fairly similar, some design patterns help realizing them.
The \gls{abstract factory} pattern proofed to be very handy for this task.
For each necessary vector (tf, idf, tf-idf) one can build a vector creator which shares the design of the other vector creators.
Therefore the abstract class \textit{VectorCreator} has been introduced.
The \textit{VectorCreator} offers the abstract method \textit{\_get\_vector(document\_id:int):DocumentVector} which will be responsible for creating all vectors.
All inherited classes will implement the abstract method with a procedure to create an instance of \textit{DocumentVector}.

\FloatBarrier

\paragraph{Term frequency vector}
\index{Term Frequency}
A term frequency vector representing a document consists of the count of a term's occurrences.
The current implementation uses the SQL query displayed in listing~\ref{lst:tf-query} for generating the tf vector.
The result of a query may look as follows (based on the tables shown in figure~\ref{fig:ertermdocumentassignment} and are displayed in table~\ref{tab:tf-query-result}.

\begin{table}

    \center

    \rowcolors{1}{\dustRowFirst}{\dustRowSecond}
    \begin{tabular}{ l|l|l }
        \rowcolor{\dustRowHead}
        \multicolumn{3}{c}{\textbf{$\text{tf}_\text{document\_1}$}}\\\hline

        term\_id & name & value \\\hline
        1   & blouse    & 1\\
        2   & blue      & 1\\
        3   & polyester & 0\\
        4   & cotton    & 1\\
        5   & green     & 0\\
        6   & trouser   & 0\\
        7   & white     & 0\\
    \end{tabular}
    \quad
    \rowcolors{1}{\dustRowFirst}{\dustRowSecond}
    \begin{tabular}{ l|l|l }
        \rowcolor{\dustRowHead}

        \multicolumn{3}{c}{\textbf{$\text{tf}_\text{document\_2}$}}\\\hline

        term\_id & name  & value\\\hline
        1    & blouse    & 1\\
        2    & blue      & 0\\
        3    & polyester & 1\\
        4    & cotton    & 0\\
        5    & green     & 0\\
        6    & trouser   & 0\\
        7    & white     & 1\\
    \end{tabular}
    \hfill\\
    \rowcolors{1}{\dustRowFirst}{\dustRowSecond}
    \begin{tabular}{ l|l|l }
        \rowcolor{\dustRowHead}
        \multicolumn{3}{ c }{\textbf{$\text{tf}_\text{document\_3}$}}\\\hline

        term\_id & name & value\\\hline
        1 & blouse  & 0\\
        2 & blue  & 0\\
        3 & polyester  & 0\\
        4 & cotton  & 1\\
        5 & green  & 1\\
        6 & trouser  & 1\\
        7 & white  & 0\\
    \end{tabular}

    \caption{Possible result of the query in listing~\ref{lst:tf-query}}
    \label{tab:tf-query-result}
\end{table}

\begin{lstlisting}[language=SQL,caption={SQL query for generating tf-vectors},label={lst:tf-query},float=h]
-- :document_id is a parameter given to the method
SELECT
    [t].[term_id]
    , [t].[name]
    , CASE WHEN  [a].[document_id] IS NULL
        THEN    0
        ELSE    [a].[count]
    END AS [value]
FROM
    [Term] AS [t]
    LEFT OUTER JOIN [TermDocumentAssigner] AS [a]
        ON  [t].[term_id] = [a].[term_id]
        AND [document_id] = :document_id
ORDER BY
    [t].[term_id]
;
\end{lstlisting}

The result of the query shown in listing~\ref{lst:tf-query} will finally be stored in the \textit{TermFrequencyVector} class derived from \textit{DocumentVector} (see figure~\ref{fig:uml-document-vectors}).

\FloatBarrier

\paragraph{Document frequency vector}
\index{Document Frequency}
In contrast to \textit{TermFrequencyVector} the \textit{DocumentFrequencyVector} resembles the whole collection of documents, rather than a single one.
Therefore the parameter \textit{document\_id} can be omitted.
But in order to sustain uniformity between all classes inheriting from \textit{VectorCreator} it will be carried along but set to a null value.
To make the source code more readable, the SQL code for querying the document-frequency values has been outsourced to a SQL-View as shown in listing~\ref{lst:df-view}.
This little tweak (which has no influence on execution-speed) left the query for df-vectors as simple as shown in listing~\ref{lst:df-query}.
The result for the example is given in table~\ref{tab:df-query-result}.

\begin{lstlisting}[language=SQL,caption={SQL statement to create the \textit{DocumentFrequency}-view},label={lst:df-view},float=h]
CREATE VIEW IF NOT EXISTS [DocumentFrequency] AS
    SELECT
            [t].[term_id]
            , [t].[name]
            , CASE WHEN   [a].[count] IS NULL
                THEN        0
                ELSE        [a].[count]
            END AS [value]
    FROM
        [Term] as [t]
        LEFT OUTER JOIN
        (
            SELECT
                [term_id]
                , SUM([document_id]) AS [count]
            FROM
                [TermDocumentAssigner]
            GROUP BY
                [term_id]
        ) AS [a]
            ON [t].[term_id] = [a].[term_id]
    ORDER BY
        [t].[term_id]
;
\end{lstlisting}

\begin{lstlisting}[language=SQL,caption={SQL query for generating df-vectors},label={lst:df-query},float=h]
SELECT
    [term_id]
    , [name]
    , [value]
FROM
    [DocumentFrequency]
;
\end{lstlisting}

\begin{table}
    \center
    \rowcolors{1}{\dustRowFirst}{\dustRowSecond}
    \begin{tabular}{ l | l | l }
        \rowcolor{\dustRowHead}
        \multicolumn{3}{ c }{\textbf{df}}\\\hline
        term\_id    & name      & value\\\hline
        1           & blouse    & 2\\
        2           & blue      & 1\\
        3           & polyester & 1\\
        4           & cotton    & 2\\
        5           & green     & 1\\
        6           & trouser   & 1\\
        7           & white     & 1\\
    \end{tabular}
    \caption{Possible results of the query in figure~\ref{lst:df-query}}
    \label{tab:df-query-result}
\end{table}

\begin{figure}[h]
    \center
    \includegraphics[scale=0.4]{inc/implementation/contentanalyzer/uml_document_vectors}
    \caption{Document vectors}
    \label{fig:uml-document-vectors}
\end{figure}

\FloatBarrier

\paragraph{Inverse document frequency vector}
\index{Inverse Document Frequency}
For building idf-vectors one can use df-vectors (and their source code) as basis.
The inverse document frequency is the logarithm of the total count of documents in a collection divided by a term's document frequency.
Another SQL-View called N-view will provide the number of documents, while the code for creating idf-vectors get outsourced into its own view once more.
Since the SQL implementation of \gls{sqlite3} does not offer a logarithm-function, one has to define his very own.
Fortunately the standard python library for connecting to sqlite3-databases supports the creation of functions as shown in listing~\ref{lst:idf-log-function}.


\begin{lstlisting}[language=Python,caption={Preparing the log-function for SQL-statement in listing~\ref{lst:idf-view}},label={lst:idf-log-function},float=h]
def _create_log_function(self, conn):
    conn.create_function('log10', 1, InverseDocumentFrequencyVectorCreator.log_10)
    pass

def log_10(x):
    base = 10
    return math.log(x, base)
\end{lstlisting}
\begin{lstlisting}[language=SQL,caption={SQL-statement to create the InverseDocumentFrequency-view},label{lst:idf-view},float=h]
CREATE VIEW IF NOT EXISTS [InverseDocumentFrequency] AS
    SELECT
        [term_id]
        , [name]
        , log10
        (
            CAST ((SELECT [document_count] from [N]) AS REAL) / [df].[value]
        )
         AS [value]
    FROM
        [DocumentFrequency] AS [df]
    ORDER BY
        [term_id]
    ;
\end{lstlisting}


\begin{lstlisting}[language=SQL,caption={SQL-statement to create the N-view},label={lst:n-view},float=h]
CREATE VIEW IF NOT EXISTS [N] AS
    SELECT
        (SELECT COUNT(*) FROM [Document]) AS [document_count]
        , (SELECT COUNT(*) FROM [Term]) AS [term_count]
;
\end{lstlisting}


\begin{lstlisting}[language=SQL,caption={SQL-query for generating idf-vectors},label={fig:idf-query},float=h]
SELECT
    [term_id]
    , [name]
    , [value]
FROM
    [InverseDocumentFrequency]
;
\end{lstlisting}


\begin{table}
    \center
    \rowcolors{1}{\dustRowFirst}{\dustRowSecond}
    \begin{tabular}{ l | l | l }
        \rowcolor{\dustRowHead}
        \multicolumn{3}{ c }{\textbf{idf}}\\\hline
        term\_id    & name      & value\\\hline
        1           & blouse    & $\log_{10}(3/2) \approx 0.18$\\
        2           & blue      & $\log_{10}(3/1) \approx 0.48$\\
        3           & polyester & $\log_{10}(3/1) \approx 0.48$\\
        4           & cotton    & $\log_{10}(3/2) \approx 0.18$\\
        5           & green     & $\log_{10}(3/1) \approx 0.48$\\
        6           & trouser   & $\log_{10}(3/1) \approx 0.48$\\
        7           & white     & $\log_{10}(3/1) \approx 0.48$\\
    \end{tabular}
    \caption{Possible results of the query in figure~\ref{fig:idf-query}}
    \label{tab:idf-query-result}
\end{table}

\FloatBarrier

\paragraph{Tf-idf vector}
\index{TfIdf}
Finally one can create tf-idf vectors which are the combination of tf- and idf-vectors (as explained in section~\ref{sec:tfidf}).
Since tf-idf is merely the multiplication of term frequency with the corresponding inverse document frequency, it is rather simple to create.
Listing~\ref{lst:tfidf-code} shows the creation of a \textit{TfIdfVector} in the method \textit{\_create\_vector()}, whereas the multiplication is in method \textit{\_get\_values()}.

\begin{lstlisting}[language=Python,caption={Python code for calculating tfidf-vectos on basis on tf- and idf-vectors},label={lst:tfidf-code},float=h]
class TfIdfVectorCreator(VectorCreator):

    def __init__(self, db_connection_str):
        super(TfIdfVectorCreator, self).__init__(db_connection_str)

        self._tf_creator = TermFrequencyVectorCreator(db_connection_str)
        self._idf_creator = InverseDocumentFrequencyVectorCreator(db_connection_str)
        pass

    def _create_vector(self, document_id):
        tf_vector = self._tf_creator.get_vector(document_id)
        idf_vector = self._idf_creator.get_vector(document_id)

        tfidf_vector = TfIdfVector()

        for triple in self._get_values(tf_vector, idf_vector):
            tfidf_vector.add_to_vector(triple)

        return tfidf_vector

    def _get_values(self, tfv, idfv):
        ingredients = zip(tfv.term_id, tfv.description, tfv.values, idfv.values)

        for (tf_tid, tf_desc, tf_val, idf_val) in ingredients:
            yield (tf_tid, tf_desc, tf_val * idf_val)
            pass
        pass
\end{lstlisting}

\begin{table}
    \center

    \rowcolors{1}{\dustRowFirst}{\dustRowSecond}
    \begin{tabular}{ l|l|l }
        \rowcolor{\dustRowHead}
        \multicolumn{3}{ c }{\textbf{$\text{tf-idf}_\text{document\_1}$}}\\\hline
        term\_id & name & value\\\hline
        1   & blouse    & $\approx 0.18$\\
        2   & blue      & $\approx 0.48$\\
        3   & polyester & 0\\
        4   & cotton    & $\approx 0.18$\\
        5   & green     & 0\\
        6   & trouser   & 0\\
        7   & white     & 0\\
    \end{tabular}
    \quad
    \rowcolors{1}{\dustRowFirst}{\dustRowSecond}
    \begin{tabular}{ l|l|l }
        \rowcolor{\dustRowHead}
        \multicolumn{3}{ c }{\textbf{$\text{tf-idf}_\text{document\_2}$}}\\\hline
        term\_id & name & value\\\hline
        1    & blouse   & $\approx 0.18$\\
        2    & blue     & 0\\
        3    & polyester& $\approx 0.48$\\
        4    & cotton   & 0\\
        5    & green    & 0\\
        6    & trouser  & 0\\
        7    & white    & $\approx 0.48$\\
    \end{tabular}

    \hfill\\

    \rowcolors{1}{\dustRowFirst}{\dustRowSecond}
    \begin{tabular}{ l|l|l }
        \rowcolor{\dustRowHead}
        \multicolumn{3}{ c }{\textbf{$\text{tf-idf}_\text{document\_3}$}}\\\hline

        term\_id & name & value\\\hline
        1        & blouse      & 0\\
        2        & blue        & 0\\
        3        & polyester   & 0\\
        4        & cotton      & $\approx 0.18$\\
        5        & green       & $\approx 0.48$\\
        6        & trouser     & $\approx 0.48$\\
        7        & white       & 0\\
    \end{tabular}


    \caption{Possible result of the function in figure~\ref{lst:tfidf-code}}
    \label{tab:tfidf-query-result}
\end{table}


With the vectors built, the main task of the content analyzer is done.
However there is still one more enhancement one can implement to make the repetitive creation of vectors faster.
Through \gls{dynamic programming} one can easily ``re-create" vectors which have already been used once.
Since it proves useful, if all inheriting classes of \textit{VectorCreator} can use dynamic programming without explicitly implementing it, one can implement the \textit{VectorCreator} as \gls{proxy}.
As a result the \textit{VectorCreator} gets another method called \textit{get\_vector(document\_id:int)} to which the same rules apply as to \textit{\_create\_vector(document\_id:int)}.
The buffering of \textit{DocumentVectors} can now be implemented in \textit{get\_vector(document\_id:int)} and all inheriting classes will also posses caching capabilities.
The code for dynamic programming is shown in listing~\ref{lst:dynamic-programming}.\\
In order to get a picture of all important vectors and their creation, a UML-diagram has been included (figure~\ref{fig:uml-vectorssimple}).

\begin{lstlisting}[language=Python,caption={Dynamic programming},label={lst:dynamic-programming},float=h]
class VectorCreator(object):

    # ... omitted unnecessary code

    def get_vector(self, document_id=None):
        if document_id is not None and not isinstance(document_id, int):
            raise TypeError('document_id must either be an integer or None')
        if not document_id in self._cache:
            vector = self._create_vector(document_id)
            if vector.document_id is None:
                vector.document_id = document_id
            self._cache[document_id] = vector
        return self._cache[document_id]}

    def _create_vector(self, document_id):
        # ... creating vector
\end{lstlisting}


\begin{figure}[h]
    \center
    \includegraphics[scale=0.33,angle=90]{inc/implementation/contentanalyzer/uml_vectors_simple}
    \caption{Abstract and concrete vector fabric}
    \label{fig:uml-vectorssimple}
\end{figure}
