

\section{Implementation}
Most of the theorie behind RS and the methodology it depends on have already been introduced.
The implementation of a RS will be described in the following.
The examples will be written in a programming language called Python.
There is also some SQL code and some UML- and EER-diagrams in order to visualize the concept.

\subsection{Content based RS in-depth}
\label{sec:implementation-contentbased}
The general framework for a content-based RS is shown in figure~\ref{fig:framework-contentbasedrs}.
There are three main components a content-based RS needs.
\begin{itemize}
    \item \textbf{Content Analyzer}\\
        Since all items the RS has to work with can potentially be unstructured, a pre-process is necessary to filter relevant information.
        This will be mainly done by techniques of IR.
        The Content Analyzer aims to bring all items in a from that can be used by its successional components.
        \citep[p.~75-77]{lops:2011}
    \item \textbf{Profile Learner}\\
        When the items are in a suitable form, the Profile Learner can construct a user profile.
        In case of Rocchio's algorithm this includes to distinguish all relevant items from non-relevant.
        With the items and the users preferences the Profile Learner can build the user profile.
        In case of Rocchio, the user profile is a vector representing his attitude towards the different attributes a item may have.
        \citep[p.~75-77]{lops:2011}
    \item \textbf{Filtering Component}\\
        For each user profile the Filtering Component can find items that may match the users preferences.
        Depending on the method implemented the result can be a binary or continuous relevance judgment.
        The continuous relevance judgement is a list of ranked items.
        \citep[p.~75-77]{lops:2011}
        The RS implemented for this thesis uses the k-nearest-neighbours (kNN) classification.
        This results in a list of ranked items where the $k$ best-ranked items will be suggested to the user.
\end{itemize}


\begin{figure}[h]
    \includegraphics[scale=0.3]{inc/rocchio/HighlevelContentBased}
    \caption{High level description of a content-based RS.\citep[p.~76]{lops:2011}}
    \label{fig:framework-contentbasedrs}
\end{figure}

\subsection{Content Analyzer}
%{Building the vectors}
%{Storage within the database}
For this project the data describing the products, offered by an online shop have been semi-structured.
It was a text file where each line described a product.
An example is given im figure~\ref{fig:productdata}.
\begin{figure}[h]
\begin{lstlisting}
ImgURL Brand Product Price Shoulder_Width Model_Length Collar_Type Material
http://i1.ztat.net/large/4E/M2/1E/00/0K/11/4EM21E000-K11@4.jpg Emoi en Plus Bluse - dazzling blue 24,95 °(\euro{})° 50 cm 70 cm bei Gr°(\"{o}\ss{})°e 44 Rundhals 100% Polyester
http://i2.ztat.net/large/NA/52/1D/03/NA/11/NA521D03N-A11@3.jpg NAF NAF WENT - Bluse - ecru/noir 38,95 °(\euro{})° 55 cm bei Gr°(\"o{}\ss{})°e S Rundhals 64% Viskose, 22% Baumwolle, 10% Modal, 4% Polyamid
\end{lstlisting}
    \caption{Example product data}
    \label{fig:productdata}
\end{figure}
Since the structure of the input was known, it was possible to filter out all relevant product information without using too fancy IR methods.
With regular expressions all relevant informations such as the product\_image-url, brand, product, price, collar type and material have been extracted and stored in a database.
Since a RS could theoretically handle any kind of item afar from products, a distinction has been made between documents in general and products.
The relation between a product and a document has been illustrated in figure~\ref{fig:ertermdocumentassignment}.
It is notable that all terms in a document are treated equally.
This means, that there are no parametric zones or such (section~\ref{sec:parametricandzoneindices}).
\begin{figure}[h]
    \center
    \includegraphics[scale=0.5]{inc/implementation/er_term_document_assignment}
    \caption{ER diagram of documents, products and associated terms}
    \label{fig:ertermdocumentassignment}
\end{figure}
Since the relation between \textit{Product} and \textit{Term} is N to N, an intermediate table is neccessary when transforming the ER-diagram into a relational model.
Therefore the table \textit{TermDocumentAssigner} will be introduced and the relational model will look as follows:

\begin{quote}
    \textbf{Document}{(\underline{document\_id}, name)}\\
    \textbf{Product}{(\underline{document\_id[Document]}, image\_name)}\\
    \textbf{Term}{(\underline{term\_id}, name)}\\
    \textbf{TermDocumentAssigner}{(\underline{document\_id[Document], term\_id[Term]})}\\
\end{quote}




\subsection{Profile Learner}
%{Rocchio algorithm}
%{Relevance feedback}

\subsection{Filtering Component}


%%%%%%%%%%%%%%%%%%%%%%%%%%
%k nearest neighbours
% manning 297, 290 (euclidean distance)
%%%%%%%%%%%%%%%%%%%%%%%%%%

\subsection{Onlineshop}
... dunno

\subsection{Testing}
wie sind die Empfehlungen des algorithmus








