
\section{Information retrieval}
%%%%%%%%%%%%%%%%%%%%%%%%%%%%%%%%%
\iffalse
Was ist Information retrieval
in depth implementation is out of scope
Hinleitung zu Vektoren
\fi
%%%%%%%%%%%%%%%%%%%%%%%%%%%%%%%%%
\citeauthor{manning:2009} describe Information Retrieval in their book as process of "finding material [\ldots] of an unstructured nature [\ldots] that satisfies an information need from within a large collection."\citep[p.~1]{manning:2009}\\
In generall IR and RS are very similiar and a few technologies and findings of IR can be transfered to RS.
Since IR systems are build to handle unstructured data and transform them into a form that is easier to handle for computer systems one can adopt the method to RS input data such as items.\citep[p.~21-23]{ricci:2011}
There are various ways IR systems interpret data.
However this thesis will focus on one common method (and the theorie it depends on) called "tf-idf-vectors" which is a requirement for the Rocchio algorithm.\citep[p.~93]{lops:2011}\\
IR systems are built to handle "unstructured data".
Unstructured data is information bundled within a document (document is the IR related term for what is understood as item by RS) without any clear semantical structure.\citep[p.~1-3]{manning:2009}
The data within a document
IR systems can extract all relevant terms from a document
A term may resemble an attribute of an item such as its colour or brand.
The way of retrieving terms from a unstructured document is out of scope for this thesis but there is a brief example in figure~\ref{fig:TermRetrieving}.\\
For the section Information Retrieval the IR vocabulary will be adopted.
This means, that items will be called documents and attributes will be known as terms.

\begin{figure}[h]
    \center
    \lstset{style=customHTML}
    \begin{lstlisting}
<html>
    <head><title>Online shop</title></head>
    <body>
        <img src="/img/p_42.jpg" alt="FancyBrand's product"/>
        <table>
            <tr><td>Colour</td><td>green</td></tr>
            <tr><td>Price</td><td>24,95 &euro;</td></tr>
            <tr><td>Brand</td><td>FancyBrand</td></tr>
        </table>
    </body>
</html>
    \end{lstlisting}
    \begin{tabular}{ l }
        \textbf{Terms:}\\\hline
        green\\
        24,95 \&euro;\\
        FancyBrand%\\\hline
    \end{tabular}
    \caption{Retrieving Terms from a HTML document.}
    \label{fig:TermRetrieving}
\end{figure}


\subsection{Weighting}
As already mentioned a document can be be described as a collection of terms.
The process of locating terms within a document is taken for granted.
When a user queries for a specific term one has to find each relevant document including this term.
Also an order regarding the documents significance will be required.
The terms significance within a document is defined by its weight.\citep[p.~117]{manning:2009}\\
While primitive IR methods such as boolean retrieval only check for the existance of a queried term within a document weighting can give more precise results regarding the terms significance in a document.\citep[p.~109]{manning:2009}

\subsubsection{Term frequency}
A simple method to quantify the importance of a term within a document is the so called term frequency.
The concept is based on the assumption that the frequency of a term $t$ indicates its importance within a document $d$.
It is denoted as $\text{tf}_{t,d}$ where $t$ resembles the term an $d$ the document.\citep[p.~117]{manning:2009}\\
For products of an online shop it may look as following.\\
\begin{figure}[h]
    \center
    Document $\text{d}1$ with identified terms underlined:\\
    \fbox{
        \parbox{\textwidth}{
            \underline{blouse} \underline{blue} \underline{55} Euro. \underline{55} cm by size \underline{S}. 100\% \underline{Polyester}
        }
    }

    \vspace{5mm}
    \rowcolors{0}{\dustRowColourFirst}{\dustRowColourSecond}
    \begin{tabular}{ l l }
        \rowcolor{\dustRowColourHead}
        \multicolumn{2}{c}{Term frequency}\\\hline
        $\text{tf}_{\text{blouse},\text{d1}}$       & 1\\
        $\text{tf}_{\text{blue},\text{d1}}$         & 1\\
        $\text{tf}_{\text{55},\text{d1}}$           & 2\\
        $\text{tf}_{\text{S},\text{d1}}$            & 1\\
        $\text{tf}_{\text{Polyester},\text{d1}}$    & 1\\
    \end{tabular}

    \caption{Term frequency weighting}
\end{figure}


\subsubsection{Document frequency}

\subsubsection{Inverse document frequency}

\subsubsection{Tf-Idf}

\subsection{Vector space model}
%%%%%%%%%%%%%%%%%%%%%%%%%%%%%%%%%
wie werden die vektoren gebaut?
%%%%%%%%%%%%%%%%%%%%%%%%%%%%%%%%%
