
\section{Introduction}

% Choice overload
% Aktuelle Situation in der Kleidungsbranche
% kurzer Einblick in Recommender systems (wird spaeter weiter ausgefuehrt)


When a typical customer enters a shop or department store, he often gets confronted with a very large amount of products he can choose from.
This is especially true, when searching for products on the internet in a web shop.
The commonly known web shop amazon.com for instance offers about 280 million different products on their internet presence for the United States.\citep{marketplaceanalytics:2014}
Even shops dedicated to only a special kind of product have a wide array of products.
On Zalando, a German online shop dedicated to clothing, one can choose out of 150.000 products of 1.500 different brands.\citep{visser:2014}\\
There is a general belief that great choice pleases the customers needs.
However it is actually proven, that this idea is not fully correct and reality is much more complex.
Already in 2000 a group of scientists demonstrated, that a great variety can also have negative effects.\citep[312]{diehl:2010}
It is proven, that large assortments can lower the customers willingness to purchase products.\citep[313]{diehl:2010}
Furthermore to much choice can also lower the customers satisfaction.\citep[320]{diehl:2010}


Since especially online-shops have a overwhelming great assortment of products, they struggle most with the problem of overloading their potential customers with products.
Therefore many online shops have already implemented recommendation systems to aid their customers by finding the right product.
Most modern recommender engines give for every user customized recommendations - these ones are also called personalized recommendation system.
But early versions also gave static recommendations that were the same for every user.\citep[1-2]{ricci:2011}




There are different possible approaches to recommendation systems and some of them will be discussed in the later sections of this thesis.






% personal vs. non-personal RS



