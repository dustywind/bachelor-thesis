

\section{Outlook}

Recommender systems seem to have a bright future.
More and more e-commerce sites use them to boost both user satisfaction and sales.
\citep[p.~4-7]{ricci:2011}
But the application of recommender systems is not restricted to websites only.
The will be further integrated into everyones life by building them into operating systems.
Ubuntu for example directs search queries to e-commerce provider such as Amazon.com to help the user finding items.\citep{ubuntu:2014}

%Future research
This thesis only focused on implementing and evaluating a RS utilizing Rocchio's algorithm.
Future works may test the combination of Rocchio's algorithm in combination with other recommender systems as this may result in good recommendations.
It would also be interesting to investigate how well Rocchio's algorithm for relevance feedback works when using it as support for a recommender system by pre-selecting items which could be relevant.
This would require to use the algorithm in its original use-case and using a search query as input.

This thesis only considered e-commerce.
However it should be possible to use some recommender techniques in real-world shops.
All technical preconditions for realizing this are already satisfied.
A content-based RS in the context of an apparel shop could be build using RFID-Tags on every piece of clothing.
Every time a user enters a changing room one could look up which clothes the user took into the cubicle and generate recommendations from this information.
All generated recommendations could be displayed on a small screen mounted in the changing cubicle.
Testing user's acceptance of RS in a physical shop and how well it works may also be an interesting field of research.

