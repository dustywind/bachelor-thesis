

\section{Recommender systems}
%%%%%%%%%%%%%%%%%%%%%%%%%%%%%%%%%%%%
Aufgaben eines Recommender systems
%%%%%%%%%%%%%%%%%%%%%%%%%%%%%%%%%%%%
Even though has already been a brief introduction into recommender systems a much more in depth view will follow.
When describing a RS, one also has to describe the different users and the goal they want to achieve.
On the one hand is the provider of an RS.
Let's assume, that the provider is a huge online shop selling apparel.
On the other hand, there is the client using the RS to find products best suited for his needs.
In case of a online shop the user probably wants clothing that is after his fancy and in his size.
The online shop however, pursues many more objectives.
When he offers a RS to the user, he wants to
\begin{itemize}
    \item increase sells
    \item increase user satisfaction to gain customer loyalty
    \item
    \item gain better understanding about the users needs
\end{itemize}
\citep[4-5]{ricci:2011}



\subsection{Different approaches}
%%%%%%%%%%%%%%%%%%%%%%%%%%%%%%%%%%%%
was fuer welche gibt es
wie unterscheiden sie sich
%%%%%%%%%%%%%%%%%%%%%%%%%%%%%%%%%%%%

\subsection{Comparison}
%%%%%%%%%%%%%%%%%%%%%%%%%%%%%%%%%%%%
vor- und nachteile
\"Ubersicht \"uber alle System (fancy Tabelle?)
%%%%%%%%%%%%%%%%%%%%%%%%%%%%%%%%%%%%






