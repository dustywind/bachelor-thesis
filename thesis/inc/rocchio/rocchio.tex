
\section{Rocchio's algorithm (theory)}
\label{sec:rocchio}
%%%%%%%%%%%%%%%%%%%%%%%%%%%%%%%%%%%%
\iffalse
Beschreibung des Rocchio algorithmus
\fi
%%%%%%%%%%%%%%%%%%%%%%%%%%%%%%%%%%%%
With section~\ref{sec:tfidf} and section~\ref{sec:vectorspacemodel} the requirements of Rocchio's algorithm have been described.\citep[p.~178]{manning:2009}
The next step is to specify the algorithm itself.
Rocchio's relevance feedback algorithm is classified as content-based RS.\citep[p.~92]{lops:2011}
The algorithm tries to find a vector representing the user that is similiar to all documents which are relevant to the user.
In addition the number of unrelevant documents matching the vector should be minimal.\citep[p.~178-181]{manning:2009}
One of the algorithms charakteristics is the way feedback (as described in section~\ref{sec:feedback}) is treated.
Every time the user gives either positive or negative feedback, the algorithm will adapt the user vector to match the new criteria.\citep[p.~92-93]{lops:2011}

\subsection{Formula}
When using Rocchio's algorithm for a recommender system one uses a user vector instead of a query vector as originally intended.
\begin{equation}
    \vec{\text{q}}_m =
        \alpha \cdot \vec{\text{q}}_0
        + \beta \cdot \frac{1}{|\text{D}_\text{r}|}\sum_{\vec{\text{d}}_j\in \text{D}_\text{r}} \vec{\text{d}}_j
        - \gamma \cdot \frac{1}{|\text{D}_\text{nr}|}\sum_{\vec{\text{d}}_j\in \text{D}_\text{nr}} \vec{\text{d}}_j
\end{equation}
